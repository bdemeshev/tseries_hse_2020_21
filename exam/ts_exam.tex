\documentclass[12pt]{article}

\usepackage{tikz} % картинки в tikz
\usepackage{microtype} % свешивание пунктуации

\usepackage{array} % для столбцов фиксированной ширины

\usepackage{indentfirst} % отступ в первом параграфе

\usepackage{sectsty} % для центрирования названий частей
\allsectionsfont{\centering}

\usepackage{amsmath, amssymb} % куча стандартных математических плюшек


\usepackage{comment}

\usepackage[top=2cm, left=1.2cm, right=1.2cm, bottom=2cm]{geometry} % размер текста на странице

\usepackage{lastpage} % чтобы узнать номер последней страницы

\usepackage{enumitem} % дополнительные плюшки для списков
%  например \begin{enumerate}[resume] позволяет продолжить нумерацию в новом списке
\usepackage{caption}


\usepackage{fancyhdr} % весёлые колонтитулы
\pagestyle{fancy}
\lhead{Временные ряды}
\chead{}
\rhead{2020-12-22, праздник временных рядов!}
\lfoot{}
\cfoot{НЕ ПАНИКОВАТЬ}
\rfoot{\thepage/\pageref{LastPage}}
\renewcommand{\headrulewidth}{0.4pt}
\renewcommand{\footrulewidth}{0.4pt}


\let\P\relax
\DeclareMathOperator{\P}{\mathbb{P}}

\usepackage{todonotes} % для вставки в документ заметок о том, что осталось сделать
% \todo{Здесь надо коэффициенты исправить}
% \missingfigure{Здесь будет Последний день Помпеи}
% \listoftodos --- печатает все поставленные \todo'шки


% более красивые таблицы
\usepackage{booktabs}
% заповеди из докупентации:
% 1. Не используйте вертикальные линни
% 2. Не используйте двойные линии
% 3. Единицы измерения - в шапку таблицы
% 4. Не сокращайте .1 вместо 0.1
% 5. Повторяющееся значение повторяйте, а не говорите "то же"



\usepackage{fontspec}
\usepackage{polyglossia}

\setmainlanguage{russian}
\setotherlanguages{english}

% download "Linux Libertine" fonts:
% http://www.linuxlibertine.org/index.php?id=91&L=1
\setmainfont{Linux Libertine O} % or Helvetica, Arial, Cambria
% why do we need \newfontfamily:
% http://tex.stackexchange.com/questions/91507/
\newfontfamily{\cyrillicfonttt}{Linux Libertine O}

\AddEnumerateCounter{\asbuk}{\russian@alph}{щ} % для списков с русскими буквами
\setlist[enumerate, 2]{label=\asbuk*),ref=\asbuk*}

%% эконометрические сокращения
\DeclareMathOperator{\Cov}{Cov}
\DeclareMathOperator{\cov}{\Cov}
\DeclareMathOperator{\plim}{plim}

\DeclareMathOperator{\Corr}{Corr}
\DeclareMathOperator{\Var}{Var}
\DeclareMathOperator{\E}{E}
\def \hb{\hat{\beta}}
\def \hs{\hat{\sigma}}
\def \htheta{\hat{\theta}}
\def \s{\sigma}
\def \hy{\hat{y}}
\def \hY{\hat{Y}}
\def \v1{\vec{1}}
\def \e{\varepsilon}
\def \he{\hat{\e}}
\def \z{z}
\def \hVar{\widehat{\Var}}
\def \hCorr{\widehat{\Corr}}
\def \hCov{\widehat{\Cov}}
\def \cN{\mathcal{N}}


\begin{document}

Дорогой храбрый воин или храбрая воительница! Удачи тебе на большом празднике по временным рядам!
Начни с того, что напиши клятву и подпишись под ней:

\vspace{10pt}
\textit{Я клянусь честью студента, что буду выполнять эту работу самостоятельно.}
\vspace{10pt}


А теперь — задачки:



\begin{enumerate}

    \item (5 баллов) Найди дискретное преобразование Фурье для последовательности $2, 2, 2, 7, 7, 7$. 
    
    \item (5 баллов) Рассмотрим уравнение $y_t = 4.25y_{t-1} - y_{t-2} + u_t$, где $(u_t)$ — белый шум, и его стационарное решение.
    

    Найди $\alpha_{-1}$, $\alpha_0$, $\alpha_1$ в представлении
    \[
        y_t = \ldots + \alpha_{-1}u_{t+1} + \alpha_0 u_t + \alpha_1 u_{t-1} + \ldots
    \]

    Хинт: $(1-4L)(1-0.25L) = 1 -4.25L + L^2$.

    
    \item (10 баллов) Перед твоими очами стационарный $ARMA(1,1)$ процесс вида
    \[
    y_t = 2 + 0.6 y_{t-1} + u_t + 0.2 u_{t-1},
    \]
    где $(u_t)$ — белый шум. 

    \begin{enumerate}
        \item Найди первые три значения автокорреляционной функции $\rho_1$, $\rho_2$, $\rho_3$.
        \item Найди первые два значения частной автокорреляционной функции $\phi_{11}$, $\phi_{22}$.
    \end{enumerate}


	\item (10 баллов) Ты бесстрашно оцениваешь $GARCH(1, 1)$ модель вида:
	\[ 
        \begin{cases}
            y_t = \sigma_t \varepsilon_t \\
            \sigma_{t+1}^{2}=\alpha_{0}+\alpha_{1} y_{t}^{2}+\beta_{1} \sigma_{t}^{2}    \\
            \varepsilon_t \sim \cN(0;1)
        \end{cases}     
    \]
	
	И получаешь оценки:
    \[ 
        \hat{\alpha}_0 = 0.1, 
    \quad \hat{\alpha}_1 = 0.3, \quad \hat{\beta}_1 = 0.7, \quad \hat{\sigma}^2_{100} = 1, \quad \hat\varepsilon_{100} = 0.5 
    \]
    
    Построй прогноз модели на один и два шага вперёд.
     
    Хинт: не забудь, что прогноз — это условное матожидание при условии доступной информации.
	
	\item (10 баллов) 
	Рассмотрим систему уравнений $VAR(1)$ модели:
	\[
	\begin{cases}
		y_t = 0.2 + 0.5 y_{t-1} + 0.1 z_{t-1} + \varepsilon_{1, t}\\
		z_t = 0.1 + 0.2 z_{t-1} +  \varepsilon_{2, t} 
	\end{cases}, \quad \Var(\varepsilon) = 9\cdot I.
\]
\begin{enumerate}
    \item Является ли система уравнений стабильной?
    \item Если да, то найди $\E(z_t)$ и $\Cov(y_t, z_t)$ у стационарного решения.
\end{enumerate}

Там ещё есть задачи, держись!!!

\newpage

\item Рассмотрим $MA(2)$ модель $y_t = u_t + \beta_1 u_{t-1} + \beta_2 u_{t-2}$.
\begin{enumerate}
    \item (5 баллов) Представь данную модель в виде модели пространства состояний (состояние-наблюдение). 
    \item (10 баллов) Максимально явно опиши все формулы пересчёта для фильтра Калмана в данной ситуации.
\end{enumerate}


\item (10 баллов) Перед твоими очами стационарный $ARMA(1,1)$ процесс вида
\[
y_t = 2 + 0.6 y_{t-1} + u_t + 0.2 u_{t-1},
\]
где $(u_t)$ — белый шум. 

Дополнительно известно, что $u_{100}=-1$, $y_{100}=4$, $\sigma^2_u = 9$. 

Построй 95\% предиктивный интервал на один и два шага вперёд.


\end{enumerate}



\end{document}