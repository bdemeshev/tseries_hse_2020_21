\documentclass[12pt]{article}

\usepackage{tikz} % картинки в tikz
\usepackage{microtype} % свешивание пунктуации

\usepackage{array} % для столбцов фиксированной ширины

\usepackage{indentfirst} % отступ в первом параграфе

\usepackage{sectsty} % для центрирования названий частей
\allsectionsfont{\centering}

\usepackage{amsmath, amssymb} % куча стандартных математических плюшек


\usepackage{comment}

\usepackage[top=2cm, left=1.2cm, right=1.2cm, bottom=2cm]{geometry} % размер текста на странице

\usepackage{lastpage} % чтобы узнать номер последней страницы

\usepackage{enumitem} % дополнительные плюшки для списков
%  например \begin{enumerate}[resume] позволяет продолжить нумерацию в новом списке
\usepackage{caption}


\usepackage{fancyhdr} % весёлые колонтитулы
\pagestyle{fancy}
\lhead{Временные ряды}
\chead{}
\rhead{2020-10-31, праздник номер раз!}
\lfoot{}
\cfoot{НЕ ПАНИКОВАТЬ}
\rfoot{\thepage/\pageref{LastPage}}
\renewcommand{\headrulewidth}{0.4pt}
\renewcommand{\footrulewidth}{0.4pt}


\let\P\relax
\DeclareMathOperator{\P}{\mathbb{P}}

\usepackage{todonotes} % для вставки в документ заметок о том, что осталось сделать
% \todo{Здесь надо коэффициенты исправить}
% \missingfigure{Здесь будет Последний день Помпеи}
% \listoftodos --- печатает все поставленные \todo'шки


% более красивые таблицы
\usepackage{booktabs}
% заповеди из докупентации:
% 1. Не используйте вертикальные линни
% 2. Не используйте двойные линии
% 3. Единицы измерения - в шапку таблицы
% 4. Не сокращайте .1 вместо 0.1
% 5. Повторяющееся значение повторяйте, а не говорите "то же"



\usepackage{fontspec}
\usepackage{polyglossia}

\setmainlanguage{russian}
\setotherlanguages{english}

% download "Linux Libertine" fonts:
% http://www.linuxlibertine.org/index.php?id=91&L=1
\setmainfont{Linux Libertine O} % or Helvetica, Arial, Cambria
% why do we need \newfontfamily:
% http://tex.stackexchange.com/questions/91507/
\newfontfamily{\cyrillicfonttt}{Linux Libertine O}

\AddEnumerateCounter{\asbuk}{\russian@alph}{щ} % для списков с русскими буквами
%\setlist[enumerate, 2]{label=\asbuk*),ref=\asbuk*}

%% эконометрические сокращения
\DeclareMathOperator{\Cov}{Cov}
\DeclareMathOperator{\Corr}{Corr}
\DeclareMathOperator{\Var}{Var}
\DeclareMathOperator{\E}{E}
\def \hb{\hat{\beta}}
\def \hs{\hat{\sigma}}
\def \htheta{\hat{\theta}}
\def \s{\sigma}
\def \hy{\hat{y}}
\def \hY{\hat{Y}}
\def \v1{\vec{1}}
\def \e{\varepsilon}
\def \he{\hat{\e}}
\def \z{z}
\def \hVar{\widehat{\Var}}
\def \hCorr{\widehat{\Corr}}
\def \hCov{\widehat{\Cov}}
\def \cN{\mathcal{N}}


\begin{document}

Дорогой храбрый воин или храбрая воительница! Удачи тебе на первом празднике по временным рядам!
Начни с того, что напиши клятву и подпишись под ней:

\vspace{10pt}
\textit{Я клянусь честью студента, что буду выполнять эту работу самостоятельно.}
\vspace{10pt}


А теперь — задачки:


\begin{enumerate}

\item Рассмотрим уравнение $y_t = 3 - 0.4y_{t-1} + u_t$, где $u_t$ независимы и нормальны $\cN(0; 9)$
Я не спрашиваю, есть ли у уравнения стационарное решение и сколько их.  
Скажу прямо: оно есть! Верь мне! И даже добавлю, что в нём $y_t$ представим в виде 
\[
	y_t = c + u_t + \alpha_1 u_{t-1} + \alpha_2 u_{t-2} + \ldots
\]

\begin{enumerate}
	\item Найди $c$ и все $\alpha_k$.
	\item Найди $\E(y_t)$, $\Var(y_t)$ и первые два значения автокорреляционной функции. 
\end{enumerate}
Дополнительно известно, что $y_{100}=5$.
\begin{enumerate}[resume]
	\item Найди 95\%-й предиктивный интервал для $y_{101}$. 
	\item Найди 95\%-й долгосрочный предиктивный интервал для $y_{100+h}$, 
	где $h \to \infty$. Зависит ли он от $y_{100}$?
\end{enumerate}
	


\item Временной ряд порождается $MA(2)$ процессом $y_t = 3 + u_t + 0.5u_{t-1} + 0.2 u_{t-3}$, 
\textcolor{red}{где $u_t$ — белый шум}.

Однако Винни-Пух строит регрессию $\hat y_t = \hat\beta_1 + \hat\beta_2 y_{t-2}$ с помощью МНК.

\begin{enumerate}
	\item Найди $\E(y_t)$, $\Var(y_t)$, $\Cov(y_t, y_s)$.
	\item Какие коэффициенты примерно получит Винни-Пух, если у него много наблюдений?
\end{enumerate}


\item Рассмотрим процесс $y_t = u_1 \sin 7t + u_2 \cos 7t$, где $u_t$ — белый шум.
\begin{enumerate}
	\item Является ли данный процесс стационарным?
	\item Можно ли представить данный процесс в виде $MA(\infty)$? 
	На всякий случай, чтобы не гуглить, я напомню, $MA(\infty)$-процесс имеет вид:

	\[
	y_t = c + \varepsilon_t + \alpha_1 \varepsilon_{t-1} + \alpha_2 \varepsilon_{t-2} + \ldots,
	\]
	где $\varepsilon_t$ — белый шум. 
	И да, обращу внимание, что шум $\varepsilon_t$ не обязательно совпадает с шумом $u_t$
\end{enumerate}

\end{enumerate}

Дорогой студент, храбро решающий онлайн контрольную! Я в тебя верю, осталось три задачи!

Смелее переходи на следующую страницу!
\newpage

\begin{enumerate}[resume]

\item У стационарного процесса $y_t$ первые две обычные корреляции равны $\rho_1 = 0.5$, $\rho_2 = 0.2$,
а ожидание равно $\E(y_t) = 20$. 

Известно, что $y_{101} = 25$, $y_{99} = 22$. Наблюдение $y_{100}$ пропущено. 
Найди наилучший точечный прогноз для $y_{100}$.


Псccст, парень! Это была задача про частные корреляции! 

\item Вспомни $ETS(AAN)$ модель, а я тебе даже уравнения напишу:

\[
\begin{cases}
y_t = \ell_{t-1} + b_{t-1} + u_t \\
\ell_t = \ell_{t-1} + b_{t-1} + \alpha u_t \\
b_t = b_{t-1} + \beta u_t \\
u_t \sim \cN(0;\sigma^2) \\
% s_t = s_{t-12} + \gamma \varepsilon_t \\
\end{cases}
\]

\begin{enumerate}
	\item Ты вчера чатик читал? Помнишь там вопрос был? Ага! 
	Докажи, что ни при каких $\ell_0$ и $b_0$ этот процесс не будет стационарным. 
	Или опровергни и приведи пример, при каких будет. 
	
	Константы $\alpha$, $\beta$ лежат в интервале $(0;1)$.
	
	\item При $l_{100} = 20$, $b_{100} = 2$, $\alpha=0.2$, $\beta=0.3$, $\sigma^2 = 25$ построй
	интервальный прогноз на один и два шага вперёд. 
\end{enumerate}



\item Величины $x_t$ равновероятно равны $0$ или $1$, 
а величины $u_t$ нормальны $\cN(0; 1)$. Все упомянутые величины независимы.
Рассмотрим процесс $z_t = x_t^2 (1-x_{t-2}) u_t$.

\begin{enumerate}
	\item Найди $\Cov(z_t, z_s)$. Стационарен ли процесс $z_t$?
	\item Скажу тебе по секрету, что $z_{100} = 2.3$. 
	Построй точечный и 95\%-й интервальный прогноз на один и два шага вперёд. 
	Чем интервальные прогнозы в этой задаче особенные?
\end{enumerate}


\end{enumerate}
\end{document}

Частичные решения

\begin{enumerate}
	\item $c=5$, $\alpha_k = 0.4^k$.
	При стремлении $h\to \infty$ для прогноза стационарного процесса становятся не 
	важны прошлые значение и интервал будет иметь вид $[\mu_y - 1.96 \sigma_y; \mu_y + 1.96 \sigma_y]$.
	\item $\E(y_t) = 3$, ковариации зануляются при $|t-s|>2$.
	При большом количестве наблюдений $\hat\beta_2 \approx \frac{\Cov(y_t, y_{t-1})}{\Var(y_t)}$.
	И $\hat\beta_1 \approx 3 - \hat\beta_2 \cdot 3$.
	\item Стационарный, $\E(y_t) = 0$, ковариации проверяются по формуле косинуса суммы.
	Представить в виде $MA(\infty)$ нельзя. 
	Доказательство такое: заметим, что наш процесс $y_t$ довольно особенный: зная $y_1$ и $y_2$ 
	можно восстановить всю траекторию процесса.
	А у $MA(\infty)$ это невозможно: например, в $y_3$ входит $\varepsilon_3$, независящий от $y_1$ и $y_2$.
	\item Находим частные корреляции решая систему Юла-Волкера.
	\item Дисперсия $b_0$ равна нулю, дисперсия $b_1$ не равна нулю, значит нестационарный. 
	\item Стационарный. Ковариации зануляются при $|t-s|>1$.
	Квадрат у $x_t$, конечно, можно убрать. 
	Особенность процесса состоит в том, что закон распределения $z_t$ не является 
	ни дискретным, ни непрерывным. 
	Если текущее значение процесса ненулевое,
	то следующее будет определённо нулевым, поэтому 
	даже 100\%-й интервал на один шаг вперёд выродится в точку.
	При прогнозе на два шага вперёд текущее значение процесса не играет роли. 
	С вероятностью $1/4$ значение будет равно нулю, поэтому 
	остаётся лишь добрать оставшуюся вероятность до 95\%.
\end{enumerate}

\end{document}
