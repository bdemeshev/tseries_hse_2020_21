\documentclass[12pt]{article}

\usepackage{tikz} % картинки в tikz
\usepackage{microtype} % свешивание пунктуации

\usepackage{array} % для столбцов фиксированной ширины

\usepackage{indentfirst} % отступ в первом параграфе

\usepackage{sectsty} % для центрирования названий частей
\allsectionsfont{\centering}

\usepackage{amsmath} % куча стандартных математических плюшек

\usepackage{comment}
\usepackage{amsfonts}

\usepackage[top=2cm, left=1cm, right=1cm, bottom=2cm]{geometry} % размер текста на странице

\usepackage{lastpage} % чтобы узнать номер последней страницы

\usepackage{enumitem} % дополнительные плюшки для списков
%  например \begin{enumerate}[resume] позволяет продолжить нумерацию в новом списке
\usepackage{caption}

\usepackage{hyperref} % гиперссылки

\usepackage{multicol} % текст в несколько столбцов


\usepackage{fancyhdr} % весёлые колонтитулы
\pagestyle{fancy}
\lhead{Домашка}
\rhead{Временные ряды}
\lfoot{}
\cfoot{}
\rfoot{}
\renewcommand{\headrulewidth}{0.4pt}
\renewcommand{\footrulewidth}{0.4pt}



\usepackage{todonotes} % для вставки в документ заметок о том, что осталось сделать
% \todo{Здесь надо коэффициенты исправить}
% \missingfigure{Здесь будет Последний день Помпеи}
% \listoftodos --- печатает все поставленные \todo'шки


% более красивые таблицы
\usepackage{booktabs}
% заповеди из докупентации:
% 1. Не используйте вертикальные линни
% 2. Не используйте двойные линии
% 3. Единицы измерения - в шапку таблицы
% 4. Не сокращайте .1 вместо 0.1
% 5. Повторяющееся значение повторяйте, а не говорите "то же"


\usepackage{fontspec}
\usepackage{polyglossia}

\setmainlanguage{russian}
\setotherlanguages{english}

% download "Linux Libertine" fonts:
% http://www.linuxlibertine.org/index.php?id=91&L=1
\setmainfont{Linux Libertine O} % or Helvetica, Arial, Cambria
% why do we need \newfontfamily:
% http://tex.stackexchange.com/questions/91507/
\newfontfamily{\cyrillicfonttt}{Linux Libertine O}

\AddEnumerateCounter{\asbuk}{\russian@alph}{щ} % для списков с русскими буквами
\setlist[enumerate, 2]{label=\asbuk*),ref=\asbuk*}
%\setlist[enumerate, 1]{label=\asbuk*),ref=\asbuk*}


%% эконометрические сокращения
\DeclareMathOperator{\Cov}{Cov}
\DeclareMathOperator*{\plim}{plim}
\DeclareMathOperator{\Corr}{Corr}
\DeclareMathOperator{\Var}{Var}
\DeclareMathOperator{\E}{E}
\let\P\relax
\DeclareMathOperator{\P}{P}
\DeclareMathOperator \hVar{\widehat{\Var}}
\DeclareMathOperator \hCorr{\widehat{\Corr}}
\DeclareMathOperator \hCov{\widehat{\Cov}}

\newcommand \hb{\hat{\beta}}
\newcommand \hs{\hat{\sigma}}
\newcommand \htheta{\hat{\theta}}
\newcommand \s{\sigma}
\newcommand \hy{\hat{y}}
\newcommand \hY{\hat{Y}}
\newcommand \e{\varepsilon}
\newcommand \he{\hat{\e}}
\newcommand \cN{\mathcal{N}}


\begin{document}

\begin{enumerate}
    \item «Взять языка». Незаметно для потенциального противника раздобудь два временных ряда: месячный и дневной. 
    Не бери цены финансовых инструментов, так как они плохо прогнозируются в силу эффективности рынка.
    Разумно взять реальные показатели. Если очень хочется работать с финансовыми данными, можно взять волатильность, 
    она прогнозируется хорошо. Максимально чётко укажи, откуда взяты ряды. Если ряды парсились, то приведи код. 
    При желании можно взять больше рядов и использовать какой-то ряд в качестве предиктора. 

    \item «Намалевич». Построй графики рядов, графики автокорреляционных функций, графики с нарезкой ряда на годы для иллюстрации сезонности. 
    
    \item «Твиттер». Кратко прокомментируй полученные графики. 
    Явлются ли ряды сезонными? есть ли тренд? стационарны ли ряды? есть ли точки излома? растёт ли амплитуда колебаний ряда?
    Возьми логарифм ряда, если душа тянется к логарифму.

    % \item «Разделяй и властвуй». Разделите ряды на обучающую и тестовую выборку в пропорции 80\% и 20\%. 

    \item «Двенадцать месяцев». Для месячного ряда сравни прогнозы разных моделей на один шаг вперёд с помощью кросс-валидации скользящим окном. 
    Длину скользящего окна выбери равной 80\% длины ряда. В качестве метрике возьми среднее абсолютное отклонение, MAE. 
    Сравни следующие модели:

    \begin{itemize}
        \item Наивная, $\hat y_{t+1} = y_t$, \verb|NaiveForecaster| из \verb|sktime|.
        \item Сезонная наивная, $\hat y_{t+1} = y_{t + 1 - 12}$, \verb|NaiveForecaster| из \verb|sktime|.
        \item SARIMA(1, 1, 1)(1, 0, 0)[12], \verb|ARIMA| из \verb|sktime|.
        \item Процедура Хиндмана-Хандакара подбора SARIMA, \verb|AutoARIMA| из \verb|sktime|.
        \item LGT из пакета \verb|orbit|.
        \item DLT из пакета \verb|orbit|.
        \item ETS(AAA), \verb|ExponentialSmoothing| из \verb|sktime|.
        \item ETS с автоматическим выбором по AIC, \verb|AutoETS| из \verb|sktime|.
        \item (по желанию) Случайный лес или градиентный бустинг на лагированных $y_{t-s}$, \verb|sktime| будет удобнее :)
    \end{itemize}

    Ссылки: 
    \begin{itemize}
        \item \verb|sktime|: \url{https://www.sktime.org}.
        \item \verb|orbit|: \url{https://orbit-ml.readthedocs.io}.
        \item Процедура Хиндмана-Хандакара, \url{https://otexts.com/fpp3/arima-r.html}.
    \end{itemize}

    Выбери наилучшую модель и построй график прогнозов для неё на один год вперёд использовав все 100\% наблюдений как обучающую выборку.

    \item «Ежедневный пророк». 
    Раздели дневной ряд на обучающую и тестовую выборку. За тестовую возьмём последние два года. Если ряд короткий, то можно и один год взять в качестве тестовой.
    Обучив модели на обучающей выборке, сравни качество прогнозов
    на тестовой выборке с помощью MAE. Сравни следующие модели:

    \begin{itemize}
        \item Наивная.
        \item Сезонная наивная.
        \item KTR из пакета \verb|orbit|.
        \item PROPHET с дефолтными настройками, \verb|sktime| удобно интергрирован с \verb|prophet|.
        \item ARIMA(1, 1, 1) с тригонометрическими предикторами, \verb|ARIMA| из \verb|sktime|.
    \end{itemize}

    Ссылки: 
    \begin{itemize}
        \item Сергей Мастицкий, Анализ временных рядов с помощью R, \url{https://ranalytics.github.io/tsa-with-r/ch-intro-to-prophet.html}.
        \item Тригонометрические предикторы, \url{https://robjhyndman.com/hyndsight/longseasonality/}.
    \end{itemize}


    Выбери наилучшую модель и построй график прогнозов для неё на один год вперёд использовав все 100\% наблюдений как обучающую выборку.


    \item (по желанию) «Атолл Бикини». Разработай и испытай свою модель для месячного ряда в STAN\footnote{Какая связь между языком STAN и атоллом Бикини? :)}.
    Сравни качество прогнозов с наивной сезонной на тестовой выборке с помощью MAE. 
    
    % \item «Итого». Сведите все MAE по всем рядам в одну табличку. Прокомментируйте, какая модель оказалась лучше для каждого ряда.


\end{enumerate}

Наставления в добрый путь храброму падавану:
\begin{itemize}
    \item Чаще используй \verb|sktime|, скорее всего там есть почти всё, что нужно :)
    \item Сдавай работу в исполняемого \verb|.ipynb| файла. Приложи \verb|.csv| файлы с рядами. 
    \item Мелкие детали, отсутствующие в условии, заполни самостоятельно, чётко описав свой выбор.
    \item Да пребудет с тобой Сила!
\end{itemize}


\end{document}
